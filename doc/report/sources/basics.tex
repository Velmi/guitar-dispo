\section{Basics}

\subsection{Digital Filters}

\subsubsection{Finite Impulse Response Filter}

Digital Filters are a huge field of digital signal processing. The most common filters are \ac{FIR}-filters.
These are also called \frqq non-recursive filters\flqq .
The basic structure is a weighted shift register without feedback. The missing feedback is what makes
the \ac{FIR}-filter stable per construction.

In \autoref{fig:FIR-filter} the structure of a \ac{FIR}-filter is shown.

\begin{figure}[!h]
    \centering
    \includegraphics[width=7cm]{img/fir.png}
    \caption{Basic structure of a $M$-th order \ac{FIR}-filter in the first canonical form \cite{meyer_signalverarbeitung}}
    \label{fig:FIR-filter}
\end{figure}

It can be seen that there are multiple components in this filter. The first component is the delay block,
which is described as $z^{-1}$. This block is responsible for adding a delay of one clock cycle to the input data.
Secondly, there are multiplication blocks, noted as $b_M$. These blocks weight the input data and give the
result to an adder, which adds $M+1$ weighted and delayed signal samples. This sum is the output of an
\ac{FIR}-filter. It also means, that the output is only dependend on the present and the $M$ last input samples
and not of the past output.

The input signal is $x$, whereas the output is $y$. The order of this Filter is $M$ which means the amount of
delay blocks.

A special case of the \ac{FIR}-filter is the \textit{Moving Average Filter}, at which all coefficients are $\frac{1}{M+1}$.

The whole system can be described with the difference equation in the time-domain (\autoref{eq:fir-difference-eq})

\begin{equation}
    y[n] = b_0 \cdot x[n] + b_1 \cdot x[n-1] + ... + b_M \cdot x[n-M]
    \label{eq:fir-difference-eq}
\end{equation},

or with the transfer function in the z-domain (\autoref{eq:fir-transfer-func})

\begin{equation}
    H(z) = b_0 + b_1 \cdot z^{-1} + ... + b_M \cdot z^{-M}
    \label{eq:fir-transfer-func}
\end{equation}.

The output can be calculated with the convolution (noted as $*$) of the impulse response $h[n]$, which is simply the
sequence of the filter coefficients, and the input signal $x[n]$ (\autoref{eq:conv}).

\begin{equation}
    y[n] = x[n] * h[n]
    \label{eq:conv}
\end{equation}

Another way to describe the filtering result is in the frequency domain.
If the spectrum of the input is given with $X(z)$ and the frequency response of the filter is $H(z)$
than the output is $Y(z)$, which can be calculated like shown in \autoref{eq:freq-response}.

\begin{equation}
    Y(z) = X(z) \cdot H(z)
    \label{eq:freq-response}
\end{equation}

The stability of the \ac{FIR}-filter is characteristic, this can be seen in the pole-zero plane. All poles are in
the center of the unit circle. This makes this filter always stable, as it is necessary that the poles of a digital
filter are in the unit circle.

If the value of the z-plane is evaluated on the unit circle, the result is the frequency response of the filter
(\autoref{fig:FIR-z-plane-and-freq}).

\begin{figure}[!h]
    \begin{subfigure}[c]{0.35\textwidth}
        \centering
        \includegraphics[width=\textwidth]{img/fir-zplane.png}
    \end{subfigure}
    \begin{subfigure}[c]{0.65\textwidth}
        \centering
        \includegraphics[width=\textwidth]{img/fir-freq.png}
    \end{subfigure}
    \caption{z-plane plot (left) and frequency response (right) of a \ac{FIR}-lowpass-filter}
    \label{fig:FIR-z-plane-and-freq}
\end{figure}

\subsubsection{Infinite Impulse Response Filter}

Another filter type is the \ac{IIR} filter. The structure is recursive as it is shown in \autoref{fig:IIR-filter}.

\begin{figure}[!h]
    \centering
    \includegraphics[width=7cm]{img/iir.png}
    \caption{Basic structure of a $M$ order \ac{IIR}-filter in the first canonical form \cite{meyer_signalverarbeitung}}
    \label{fig:IIR-filter}
\end{figure}

Due to this recursive structure, this type of filter is no longer necessarilly stable. This means that it can
oscillate and possibly the output can grow exponentially. To fulfill stability, all the poles of the
\ac{IIR}-filter have to be in the unit-circle \cite{meyer_signalverarbeitung}.

Its transfer function can be described as follows:

\begin{align}
    H(z) = \frac{1}{a_1 \cdot z^{-1} + ... + a_M \cdot z^{-M}}
\end{align}

In the time domain, the output is equal to

\begin{align}
    y[n] = x[n] - a_1 \cdot y[n-1] - a_2 \cdot y[n-2] - ... - a_M \cdot y[n-M]
\end{align}
.

The output is dependend on the current signal and all the $M$ outputs before. In literature and other documentation,
the $a_0$-coefficient is also referenced. This is the amplification of the output signal. Due to filter-stability-reasons,
this coefficient is typically set to $o$ \cite{arm_dsp} \cite{cookbook_audio}.

If the two types of filters are getting merged, the result is also an \ac{IIR}-filter, plus
a few additional feed-forward-terms. A widely used example of this is the \textit{Biquad-filter} \cite{arm_dsp}.

These filters use three filter-taps in the forward- and two taps in the feedback-path. The structure
is shown in \autoref{fig:biquad-structure}.

\begin{figure}[!h]
    \centering
    \includegraphics[width=7cm]{img/biquad_structure.png}
    \caption{Structure of a biquad filter \cite{arm_dsp}}
    \label{fig:biquad-structure}
\end{figure}

The transfer function of this filtertype is:

\begin{align}
    H(z) = \frac{b_0 + b_1 \cdot z^{-1} + b_2 \cdot z^{-2}}{a_1 \cdot z^{-1} + a_2 \cdot z^{-2}}
\end{align}

Here it can be seen, why this filter is called \frqq biquad\flqq{}: it is the short form for
\frqq bi-quadratic\flqq{}, which relates to the powers of the numerator- and denominator-terms.
Also in this structure, the $a_0$-coefficient is set to $0$.

This transfer function leeds to an output of:

\begin{align}
    y[n] = b_0 \cdot x[n] + b_1 \cdot x[n-1] + b_2 \cdot x[n-2] - a_1 \cdot y[n-1] - a_2 \cdot y[n-2]
\end{align}

It is worth noting, that the denominator-coefficients have to be negated to match the structure in
\autoref{fig:biquad-structure}.


\subsubsection{Filter Design and comparison}

This section deals with designing filters. There are a few ways to calculate the numerator and
denominator-coefficients. But this shouldn't be part of this section, the goal is more like looking at
how to design a filter in a high-level perspective.

First of all, the number of filtertaps will be discussed. As stated earlier, \ac{FIR}-filters have
the convenient property that they are always stable. The downside of these filters are,
that for example a narrow-band bandpass filter needs more filtertaps, than a comparable
\ac{IIR}-filter \cite{meyer_signalverarbeitung}. This should not be a huge problem in the most cases,
but if the goal is to variate the filter-coefficients, this may lead to problems. Additionally, a longer filter
could lead to a bigger latency.

Secondly, the shape of the frequency responses of the \ac{FIR} and \ac{IIR}-filters differ. The recursive filter
is more often used to model an actual analog filter \cite{meyer_signalverarbeitung}.

In order to compare these two types of filters, two bandpass-prototypes are being simulated. Given
is a 3\,dB-bandwidth at a certain frequency which should be achieved by both filters. The number of necessary
coefficients for this frequency response is a metric which can be used to compare the \ac{FIR}-
with the \ac{IIR}-filter.

The bandwidth is defined as

\begin{align}
    BW = f_{3dB} = f_{high} - f_{low}
\end{align}

and it can be used to calculate the quality-factor of the filter:

\begin{align}
    Q = \frac{f_{res}}{BW}
    \label{eq:q-factor}
\end{align}

Given is a bandwidth of 1\,kHz at the mid-frequency of 5\,kHz. This leads to a quality-factor of 5.

% To compare these two types of filters, two bandpass-prototypes are being simulated.
% All the parameters of these two filters, mid-frequency $f_{res}$, bandwidth $BW$, quality-factor $Q$ and
% number of coefficients are set equal in order to make them comparable.

For the generation of these filters, the python package \frqq scipy\flqq{} with the module \frqq signal\flqq{} was used
\cite{scipy}.

For the \ac{FIR}-Filter 85 coefficients were necessary to achieve these criteria, while the
\ac{IIR}-Filter just needs 6 coefficients. This is due to the fact, that \ac{IIR}-filters have poles, which
can produce sharper edges, but it can also lead to instability. This makes the \ac{IIR}-Filter more suitable,
if the number of coefficients is the only criterion.

If the frequency response of the filter is evaluated, it can be observed that the 3\,dB bandwidth
is not the only difference. Also the shape of these filters differ. This may be important for some
applications, for this specific problem, is has to be evaluated.

In the following graphics, there is the \ac{FIR}- on the left and the \ac{IIR}-bandpass-filter frequency
response on the right.

\begin{figure}[!h]
    \centering
    \begin{subfigure}[c]{0.49\textwidth}
        \centering
        \includegraphics[width=\textwidth]{img/fir_bandpass.png}
    \end{subfigure}
    \begin{subfigure}[c]{0.49\textwidth}
        \centering
        \includegraphics[width=\textwidth]{img/iir_bandpass.png}
    \end{subfigure}
    \caption{Comparison of the frequency response of a \ac{FIR}- and \ac{IIR}-bandpass-filter}
    \label{fig:fir-iir-compare}
\end{figure}

The red dots mark the desired bandwidth.

It can be seen, that the frequency response of the \ac{FIR}-filter has a convex form, whereas the frequency response
of the \ac{IIR}-filter is concave in shape.

It is worth noting that by changing the mid-frequency and holding the quality-factor constant, the bandwidth also
changes proportionally (see \autoref{eq:q-factor}). By decreasing the frequency the \ac{FIR}-filter is not capable
of achieving this bandwidth if the number of coefficients do not increase. This is not true for the \ac{IIR}-filter,
the bandwidth can always be achieved. This is shown in \autoref{fig:fir-iir-dif_freq-compare}.

\begin{figure}[!h]
    \centering
    \begin{subfigure}[c]{0.49\textwidth}
        \centering
        \includegraphics[width=\textwidth]{img/fir_bandpass_compare.png}
    \end{subfigure}
    \begin{subfigure}[c]{0.49\textwidth}
        \centering
        \includegraphics[width=\textwidth]{img/iir_bandpass_compare.png}
    \end{subfigure}
    \caption{Comparison of the frequency response of a \ac{FIR}- and \ac{IIR}-bandpass-filter at different mid-frequencies}
    \label{fig:fir-iir-dif_freq-compare}
\end{figure}

As a reference, an analog bandpass-filter is shown in \autoref{fig:analog-bandpass}, designed with the Texas Instruments
filterdesign-tool \cite{TI-filterdesign-tool}.
The parameters the tool gets are also a bandwidth of 1\,kHz at a mid-frequency of 5\,kHz.

\begin{figure}[!h]
    \centering
    \includegraphics[width=8cm]{img/analog_bandpass_ti.png}
    \caption{Analog bandpass filter frequency response}
    \label{fig:analog-bandpass}
\end{figure}

The tool calculated a 6-th order filter with the given filtertype being \frqq Bessel\flqq. This type was also used to
calculate the \ac{IIR}-filter coefficients.

It can be seen, that the basic shape of the \ac{IIR}-filter is much closer to the analog equivalent than the
\ac{FIR}-filter.

\subsection{Microcontroller-Peripherals}

\subsubsection{Inter-Integrated Sound}

The \ac{I2S} protocol was developed by \textit{Philips} to share audio between \acp{IC}.
A similarity to the \ac{SPI} protocol can be recognized.

There are three signal lines:
\begin{itemize}
    \item SCK: Clock signal
    \item SD: Data signal
    \item WS: Word select, for distinction between left and right channel
\end{itemize}
So the data is transmitted over the same signal line by time division multiplexing. 
To illsutrate the timing of the protocol the corresponding diagrams are shown in \autoref{fig:i2s-timing} \cite{nxp_i2s}.

\begin{figure}[!h]
    \centering
    \includegraphics[width=10cm]{img/i2s_timing.png}
    \caption{Timing diagramm of the \ac{I2S} protocol \cite{nxp_i2s}}
    \label{fig:i2s-timing}
\end{figure}

The word length can vary between transmitter and receiver. That is why the \ac{MSB} is sent first in the standard 
configuration. Additionally the participants do not need to know the word length of the counterpart
in advance \cite{nxp_i2s}.

\begin{figure}[!h]
    \centering
    \includegraphics[width=14cm]{img/i2s_config.png}
    \caption{The three basic configurations of the \ac{I2S} protocol \cite{nxp_i2s}}
    \label{fig:i2s-config}
\end{figure}

The wiring of this serial bus is possible in three basic configurations (shown in \autoref{fig:i2s-config}).
Thereby the controller is always responsible for the SCK and the WS line.

\subsubsection{Direct Memory Access}
\label{sec:dma}

\ac{DMA} is a feature in computer systems that allows certain hardware components or peripheral devices to directly
access the system's main memory independently of the \ac{CPU}. This enables data to be transferred
between memory and peripherals, such as \acp{ADC} or \ac{I2S}, without requiring constant \ac{CPU} intervention.
As a result, the \ac{CPU} is freed up to perform other tasks while the data transfer occurs in parallel, improving
overall system efficiency and performance \cite{rm0468}.

The implementation of DMA hardware varies depending on the microcontroller model. In the STM32H7 series, multiple \ac{DMA}
channels are available, allowing multiple data streams to be handled simultaneously by a single \ac{DMA} controller. There
are also two operational odes: normal mode and circular mode. In normal mode, each data transfer must be manually
restarted after the previous transfer completes. Conversely, in circular mode, the transfer requires only a single
initialization, as it continuously cycles through the buffer, automatically reloading data as needed \cite{rm0468}.

This setup enhances efficiency for continuous data operations, such as audio processing or real-time sensor data
acquisition, as it minimizes CPU intervention and supports uninterrupted data flow \cite{rm0468}.

For this task it is advantageous, because the signal processing can be done while the data transfer is managed
by the \ac{DMA}.
