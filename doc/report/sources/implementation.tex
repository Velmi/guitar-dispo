\section{Implementation}

\subsection{Introduction of the systems used}

For the first implementation of basic filtering operations, the evaluation board \textit{Nucleo-H743ZI} of STMicroelectronics
is used. The breakout board \textit{PmodI2S Stereo IN/OUT} is used in order to sample audio data from a 3,5\,mm jack input.

The choice for the \ac{IDE} fell on the
\newline \textit{STM32CubeIDE} as it provides an easy to use installation and
development process.

Digital filters need filter coefficients. As it would be too time consuming to calculate them during runtime, these
coefficients where computed beforehand and then stored in the memory of the microcontroller. For these computations,
own Python scripts were developed.

In order to design the \ac{PCB}, the \ac{EDA}-software \textit{KiCAD} was used.

\subsection{Implementation of the filters}

As the goal is to develop a variable bandpass filter, not only a set of coefficients for one filter has to be stored,
but many versions of this filter with different mid-frequencies.

As a reference, the frequency response of \textit{Cry-Baby}-Pedal of Prof. Litzenburger has been measured, this is
shown in \autoref{fig:wahwah-filter}.

\begin{figure}[!h]
    \centering
    \begin{subfigure}[c]{0.45\textwidth}
        \centering
        \includegraphics[width=\textwidth]{img/wahwah-1.png}
    \end{subfigure}
    \begin{subfigure}[c]{0.45\textwidth}
        \centering
        \includegraphics[width=\textwidth]{img/wahwah-2.png}
    \end{subfigure}
    \caption{Measurement of the frequency response of a Wah-Wah-pedal at the lowest and highest frequencies}
    \label{fig:wahwah-filter}
\end{figure}

Firstly, the resonance frequency varies between 466\,Hz and 2,3\,kHz.

It is also to be seen, that as the frequency changes, the bandwidth also seem to change. Firstly, it has to be mentioned,
that this is a logarithmic scale. Nevertheless, in an ideal bandpass-filter, the quality factor is constant. This factor indicates
how much energy is dissipated, or how high the loss of a oscillation circuit is. So to keep the Q-factor constant, with
increasing the resonance frequency, the bandwidth also have to increase (\autoref{eq:q-factor}).

\begin{align}
    Q = \frac{f_{res}}{f_{3\,dB}}
    \label{eq:q-factor}
\end{align}

By measuring the bandwidths of this filter, it turns out, that the quality factor also changes.

The bandwidth at 466\,Hz is measured to be 54\,Hz, which leads to a Q-factor of:

\begin{align}
    Q = \frac{460\,Hz}{54\,Hz} = 8,518
\end{align}
.

At 2,3\,kHz the bandwidth is 600\,Hz resulting in a quality factor of:

\begin{align}
    Q = \frac{2,3\,kHz}{600\,Hz} = 3,833
\end{align}
.

This behaviour is directly related to how the effect sounds and should therefore be considered in the filter design.

For the first tries, \ac{FIR}-filters were used, because they always guarantee stability \cite{meyer_signalverarbeitung}.
The first prototype of the coefficients combine a set of 32 coefficient sets with the following parameters:

\begin{table}[!h]
    \centering
    \caption{Parameters for \ac{FIR}-filterdesign}
    \label{table:fir-filterdesign}
    \begin{tabular}{c | c }
        quality factor & number of coefficients\\
        \hline
        6 & 130
    \end{tabular}
\end{table}

Here, the Q-factor is held constant for the first experiments.

In the following picture shows a \ac{FIR}-model of the analog reference at 460\,Hz and 2242\,Hz
\autoref{fig:wahwah-filter}.

\begin{figure}[!h]
    \centering
    \begin{subfigure}[c]{0.49\textwidth}
        \centering
        \includegraphics[width=\textwidth]{img/fir_bandpass460.png}
    \end{subfigure}
    \begin{subfigure}[c]{0.49\textwidth}
        \centering
        \includegraphics[width=\textwidth]{img/fir_bandpass2242.png}
    \end{subfigure}
    \caption{Measurement of the \ac{FIR}-model frequency response at the lowest and highest frequencies}
    \label{fig:wahwah-fir}
\end{figure}

It can be clearly observed, that the shape of the filters are not close to the analog reference.
The effect of the bandwidth getting wider is not visible.

The second approach to model the filter is by using a biquad-filter \cite{arm_dsp}.
These filters use two filter-taps in the forward- and two taps in the feedback-path. The structure
is shown in \autoref{fig:biquad-structure}.

\begin{figure}[!h]
    \centering
    \includegraphics[width=7cm]{img/biquad_structure.png}
    \caption{Structure of a biquad filter \cite{arm_dsp}}
    \label{fig:biquad-structure}
\end{figure}

The transfer function of this filtertype is:

\begin{align}
    H(z) = \frac{b_0 + b_1 \cdot z^{-1} + b_2 \cdot z^{-2}}{a_1 \cdot z^{-1} + a_2 \cdot z^{-2}}
\end{align}

Here it can be seen, why this filter is called \frqq biquad\flqq{}: it is the short form for
\frqq bi-quadratic\flqq{}, which relates to the powers of the numerator- and denominator-terms.
The $a_0$-coefficient is mostly set to $1$, because its meaning is the same as the $b_0$-coefficient.

Designing a bandpass-filter with a biquad-filter results in the filters shown in \autoref{fig:wahwah-iir}.

\begin{figure}[!h]
    \centering
    \begin{subfigure}[c]{0.49\textwidth}
        \centering
        \includegraphics[width=\textwidth]{img/iir_bandpass460.png}
    \end{subfigure}
    \begin{subfigure}[c]{0.49\textwidth}
        \centering
        \includegraphics[width=\textwidth]{img/iir_bandpass2242.png}
    \end{subfigure}
    \caption{Measurement of the \ac{IIR}-model frequency response at the lowest and highest frequencies}
    \label{fig:wahwah-iir}
\end{figure}

It shows, that the shape of this filter matches the analog reference better than the \ac{FIR}-filter.

\subsection{Evaluation of the filters}

Firstly the \ac{FIR}-filter is evaluated.

The first issue that will be recognized, is the cracking noise, when the filter coefficients are being
updated.

%// TODO: measure, sound of filter

\subsection{Development of the PCB}

In order to get a handy pedal, the whole hardware should be integrated into one single \ac{PCB}.
The dimensions of this should not exceed the standard-sized size of effect pedals. As a reference the
\textit{Boss DS-1} distortion pedal were used to derive a reasonable size for the \ac{PCB}. The dimensions of the
distortion pedal are 73 x 129\,mm, which should not be exceeded. The final \ac{PCB} now has dimensions of
56,39 x 84,33\,mm.

Furthermore, there should be enough inputs, to provide to ability to extend the functionality of this board.
The idea is, to develop a platform, where any effect can be implemented. In order to fulfill this, three analog and
four digital inputs are connected to solderpads for later use. Additionally, an \ac{I2C} connector is also
provided, if for example a little display should be used.

The main components that were chosen are:

\begin{itemize}
    \item STM32H725 microcontroller
    \item CS5343-CZZ Audio \ac{DAC}
    \item CS4344-CZZR Audio \ac{ADC}
\end{itemize}

The reason for the choice of these components is, that the these are also present on the Nucleo- and the
breakout-board. Except for the microcontroller, therefore the smallest package with 68 pins were chosen.

\subsection{Evaluation of the hardware}

There are a few issues on the board, which have been found and fixed so far.

\textbf{Issue 1:}

The first issue is, that the core of the microcontroller is not connected to a source. This is, because
the voltage regulator of this package is not connected to the core. This must be done 
externally. To fix this, a switched-mode power supply was added to the board, which outputs the 1\,V
for the core.

\textbf{Issue 2:}

The output voltage of the extra added power supply is noisy, which led to connection errorswith the debugger.
In order to fix this, extra capacitors were added to the supply of the core.

\textbf{Current state:}

It is possible to connect a J-Link debugger to the microcontroller.
